\documentclass[a4paper, 11pt]{article}
\usepackage[top=3cm, bottom=3cm, left = 2cm, right = 2cm]{geometry} 
\geometry{a4paper} 
\usepackage[utf8]{inputenc}
\usepackage{textcomp}
\usepackage{graphicx} 
\usepackage{amsmath,amssymb}  
\usepackage{bm}  
\usepackage[pdftex,bookmarks,colorlinks,breaklinks]{hyperref}  
\usepackage{listings}
\hypersetup{linkcolor=black,citecolor=black}
%\hypersetup{linkcolor=black,citecolor=black,filecolor=black,urlcolor=black} % black links, for printed output
\usepackage{memhfixc} 
\usepackage{pdfsync}  
\usepackage{fancyhdr}
\pagestyle{fancy}

\title{Building a Spotify Clone}
\author{David Márquez Mínguez}

\begin{document}
\maketitle
\tableofcontents

\pagebreak
\section{Introducción}

\subsection{Contexto}

Este documento alberga todos los problemas que se me han planteado a la hora de construir un clon de Spotify utilizando react-native. Además de ello, se incluirá en el mismo
los pasos realizados y las decisiones tomadas para cada uno de los desarrollos.

\subsection{Motivación}

La motivación principal para empezar el proyecto es aprender y mejorar mis habilidades como programador en el desarrollo de aplicaciones.

\subsection{Objectivos y limitaciones}

Puesto que aún no soy un experto en este campo, seguro que surgirán errores en el análisis y desarrollo de las soluciones propuestas. El objetivo de este documento no solo se 
trata de aprender, sino de dejar evidenciado cada una de las decisiones tomadas y mejorar al respecto.


\pagebreak
\section{Configuración del entorno de desarrollo}

Antes de comenzar el proyecto, el primer paso es la configuración del entorno. Para ello accedemos a la página oficial de react-native y seguimos los pasos 
\cite{setup-development-environment}. En mi caso, puesto que ya he trabajado con alguna aplicación móvil pequeña, no será necesaria mucha modificación.

Algo que si he tenido en cuenta y es importante comentar, es la eliminación del paquete \emph{@react-native-community/cli}. La documentación aconseja su eliminación, 
ya que pueden surgir problemas inesperados, para ello ejecuto:

\begin{lstlisting}
    react-native-cli @react-native-community/cli
\end{lstlisting}

Antes e iniciar el proyecto, algo que es importante comentar es que React Native tiene una interfaz de línea de comandos incorporada, esto significa que se puede generar 
un nuevo proyecto utilizando simplemente:

\begin{lstlisting}
    npx react-native@latest init src
\end{lstlisting}

Una vez iniciado el proyecto y con todos los paquetes necesarios instalados, precedemos a comprobar que la ejecución funciona correctamente. Para ello nos creamos un 
dispositivo móvil virtual utilizando Android Studio. Una vez realizados todos los pasos ejecutamos los comandos:

\begin{lstlisting}
    npm start
    npm run android
\end{lstlisting}

El primero inicia Metro, un paquete de JavaScript que se incluye con React Native. Este devuelve un solo archivo JavaScript que incluye todo el código con sus dependencias.
El segundo instala e inicia la aplicación en este caso en Android.





















\pagebreak
\section{System Design}

My system worked as follows \ldots

\pagebreak

\section{Evaluation}

We did some experiments \ldots

\pagebreak

\section{Conclusions and Future Work}

From our experiments we can conclude that \ldots

\begin{figure}[tphb]
    \centering
    \includegraphics[width=7in]{img/testimg.png}
    \caption{testimg}
    \label{img:testimg}
\end{figure}

\pagebreak

\bibliographystyle{abbrv}
\bibliography{references}  % need to put bibtex references in references.bib 
\end{document}
